\documentclass{article}

\usepackage{fancyhdr}
\usepackage{extramarks}
\usepackage{amsmath}
\usepackage{amsthm}
\usepackage{amsfonts}
\usepackage{tikz}
\usepackage[plain]{algorithm}
\usepackage{algpseudocode}
\usepackage{xcolor}
\usepackage{enumerate}
\usepackage{amssymb}
\usepackage{graphicx}
\usepackage{float} 
\usepackage{subfigure}
\usepackage[utf8]{inputenc}


\usetikzlibrary{automata,positioning}

\topmargin=-0.45in
\evensidemargin=0in
\oddsidemargin=0in
\textwidth=6.5in
\textheight=9.0in
\headsep=0.25in

\linespread{1.1}

\pagestyle{fancy}
\lhead{\hmwkAuthorName}
\chead{\hmwkClass\ (\hmwkClassInstructor\ \hmwkClassTime): \hmwkTitle}
\rhead{\firstxmark}
\lfoot{\lastxmark}
\cfoot{\thepage}

\renewcommand\headrulewidth{0.4pt}
\renewcommand\footrulewidth{0.4pt}

\setlength\parindent{0pt}

\newcommand{\enterProblemHeader}[1]{
    \nobreak\extramarks{}{Problem \arabic{#1} continued on next page\ldots}\nobreak{}
    \nobreak\extramarks{Problem \arabic{#1} (continued)}{Problem \arabic{#1} continued on next page\ldots}\nobreak{}
}

\newcommand{\exitProblemHeader}[1]{
    \nobreak\extramarks{Problem \arabic{#1} (continued)}{Problem \arabic{#1} continued on next page\ldots}\nobreak{}
    \stepcounter{#1}
    \nobreak\extramarks{Problem \arabic{#1}}{}\nobreak{}
}

\setcounter{secnumdepth}{0}
\newcounter{partCounter}
\newcounter{homeworkProblemCounter}
\setcounter{homeworkProblemCounter}{5}
\nobreak\extramarks{Problem \arabic{homeworkProblemCounter}}{}\nobreak{}

\newenvironment{homeworkProblem}[1][-1]{
    \ifnum#1>0
        \setcounter{homeworkProblemCounter}{#1}
    \fi
    \section{Problem \arabic{homeworkProblemCounter}}
    \setcounter{partCounter}{1}
    \enterProblemHeader{homeworkProblemCounter}
}{
    \exitProblemHeader{homeworkProblemCounter}
}

\newcommand{\hmwkTitle}{HW\ \#11}
\newcommand{\hmwkDueDate}{September 22, 2023}
\newcommand{\hmwkClass}{Tandon CS Bridge}
\newcommand{\hmwkClassTime}{Extended 24-week}
\newcommand{\hmwkClassInstructor}{Ratan Dey}
\newcommand{\hmwkAuthorName}{\textbf{Kunhua Huang}}
\newcommand{\hmwkNYUID}{kh4092}

\title{
    \vspace{2in}
    \textmd{\textbf{\hmwkClass:\ \hmwkTitle}}\\
    \normalsize\vspace{0.1in}\small{Due\ on\ \hmwkDueDate}\\
    \vspace{0.1in}\large{\textit{\hmwkClassInstructor\, \hmwkClassTime}}
    \vspace{3in}
}

\author{
    \vspace{0.1in}
    \textmd{\hmwkAuthorName}\\
    \text{\hmwkNYUID}
}
\date{}

\renewcommand{\part}[1]{\textbf{\large Part \Alph{partCounter}}\stepcounter{partCounter}\\}

%
% Various Helper Commands
%

% Useful for algorithms
\newcommand{\alg}[1]{\textsc{\bfseries \footnotesize #1}}

% For derivatives
\newcommand{\deriv}[1]{\frac{\mathrm{d}}{\mathrm{d}x} (#1)}

% For partial derivatives
\newcommand{\pderiv}[2]{\frac{\partial}{\partial #1} (#2)}

% Integral dx
\newcommand{\dx}{\mathrm{d}x}

% Alias for the Solution section header
\newcommand{\solution}{\textbf{\large Solution}}

\newcommand\logeq{\mathrel{\raisebox{.66pt}{:}}\Leftrightarrow}

% Probability commands: Expectation, Variance, Covariance, Bias
\newcommand{\E}{\mathrm{E}}
\newcommand{\Var}{\mathrm{Var}}
\newcommand{\Cov}{\mathrm{Cov}}
\newcommand{\Bias}{\mathrm{Bias}}

\begin{document}

\maketitle

\pagebreak

\begin{homeworkProblem}
\begin{enumerate}
    \item     Use mathematical induction to prove that for any positive integer $n$, $3$ divide $n^3 + 2n$.
    \\
    \textbf{Solution}
    \\1. \textbf{Base case}: When $n = 1$, $n^3 + 2n = 3$, which can be divided by 3.\\
    2. \textbf{Inductive step}: Assume that when $n = k$, where $k$ is some integer, $3$ divides $n^3 + 2n = k^3 + 2k$.\\
    When $n = k + 1$, $n^3 + 2n = (k + 1)^3 + 2(k + 1) = k^3 + 3k^2+3k+1+2k+2 = k^3+2k+3(k^2+k+3)$.\\
    Because $3$ divides $3(k^2+k+3)$ and $k^3 + 2k$ by the induction hypothesis. So, $3$ divides $n^3+2n$ when $n=k+1$.\\
    Therefore, for any positive integer $n$, $3$ divides $n^3+2n$.
    \item Use strong induction to prove that any positive integer $n(n\geq2)$ can be written as a product of primes.
    \\
    \textbf{Solution}
    \\1. \textbf{Base case}: When $n = 2$, it is a product of the prime number $2$ of itself.\\
    2. \textbf{Inductive step}: Assume that for $k \geq 2$, any integer $j \in [2, k]$ can be expressed as a product of prime numbers.
    If $k + 1$ is prime, then it is already a product of the prime number of itself.\\
    If $k + 1$ is not prime, it can be expressed as the product of two integer $a$ and $b$, which are both greater than 2.\\
    $k+1=a\times b \leftrightarrow a = (k+1) / b$. Because $b$ is greater than $2$, $a = (k+1)/b<k+1$. Then $a \leq k$.\\
    Symmetrically we get $b \leq k$. Thus, $a$ and $b$ can be expressed as $a=p_1\cdot p_2 \cdots p_m$ and $b=q_1\cdot q_2\cdots q_n$, respectively.\\
    Therefore, $k + 1$ can be expressed as a product of primes $k+1=a\cdot b = (p_1\cdot p_2 \cdots p_m)\cdot q_1\cdot q_2\cdots q_n$
\end{enumerate}
 \end{homeworkProblem}

\pagebreak

\begin{homeworkProblem}
    Solve the following questions from the Discrete Math ZyBook:
    \begin{enumerate}
        \item Define $P(n)$ to be the assertion that: $\sum_{j = 1}^{n} j^2 = \frac{n(n+1)(2n+1)}{6}$.
        \textbf{Solution}
        \begin{enumerate}
            \item $P(3) = \sum_{j = 1}^{3} j^2 = 14 = \frac{3(3+1)(2\cdot 3+1)}{6}$
            \item $\sum_{j = 1}^{k} j^2 = \frac{k(k+1)(2k+1)}{6}$
            \item $\sum_{j = 1}^{k + 1} j^2 = \frac{(k+1)(k+2)(2k+3)}{6}$
            \item In the base case, we need to prove that when $n = 1$, $\sum_{j = 1}^{n} j^2 = \frac{n(n+1)(2n+1)}{6}$
            \item In the inductive step, we need to prove that when $n = k+1$, $\sum_{j = 1}^{n} j^2 = \frac{n(n+1)(2n+1)}{6}$
            \item Inductive hypothesis: when $n = k$, $\sum_{j = 1}^{n} j^2 = \frac{n(n+1)(2n+1)}{6}$.
            \item $\sum_{j = 1}^{k + 1} j^2 = \sum_{j = 1}^{k} j^2 + (k + 1)^2 = \frac{k(k+1)(2k+1)}{6} + (k + 1)^2$
            \\
            \\
            $= \frac{(k + 1)[k(2k+1)+(k+1)]}{6} = \frac{(k + 1)(k + 2)(k + 3)}{6}$
            Therefore, $\sum_{j = 1}^{n} j^2 = \frac{n(n+1)(2n+1)}{6}$
        \end{enumerate}
        \item Prove that for $n \geq 1, \sum_{j = 1}^{n} \frac{1}{j^2} \leq 2 - \frac{1}{n}$
        \textbf{Solution}
        \begin{proof}
            1. \textbf{Base case}: When $n = 1$, $\sum_{j = 1}^{n} \frac{1}{j^2} = 1 = 2 - \frac{1}{n}$\\
            2. \textbf{Inductive step}: Assume that for $n = k, \sum_{j = 1}^{n} \frac{1}{j^2} \leq 2 - \frac{1}{n} = 2 - \frac{1}{k}$\\
            When $n = k + 1, \sum_{j = 1}^{k + 1} \frac{1}{j^2} = \sum_{j = 1}^{k} \frac{1}{j^2} + \frac{1}{(k + 1)^2} \leq 2 - \frac{1}{k} + \frac{1}{(k + 1)^2}$\\
            $\because \frac{(\frac{1}{k} - \frac{1}{(k + 1)^2})}{\frac{1}{(k + 1)}} = 1+\frac{k + 1}{k} - \frac{1}{k + 1} = 1 + \frac{k^2+k+1}{k^2+k} > 1$\\
            $\therefore \frac{1}{k} - \frac{1}{(k + 1)^2} > \frac{1}{k + 1}$\\
            $\therefore 2 - \frac{1}{k} + \frac{1}{(k + 1)^2} < 2 - \frac{1}{k + 1}$\\
            $\therefore \sum_{j = 1}^{k + 1} \frac{1}{j^2} \leq 2 - \frac{1}{k} + \frac{1}{(k + 1)^2} \leq 2 - \frac{1}{k + 1}$\\
            Therefore, for $n \geq 1, \sum_{j = 1}^{n} \frac{1}{j^2} \leq 2 - \frac{1}{n}$
        \end{proof}
        \item Prove that for any positive integer $n$, $4$ evenly divides $3^{2n}-1$.
        \\
        \begin{proof}
            1. \textbf{Base case}: When $n = 1$, $4$ evenly divides $3^{2\cdot 1} - 1 = 8$.\\
            2. \textbf{Inductive step}: Assume that for $n = k$, $4$ divides $3^{2\cdot k} - 1$\\
            When $n = k + 1, 3^{2\cdot(k+1)}-1=3^{2k+2}-1=3^2\cdot 3^{2k}-3^2+3^2-1=3^2(3^{2k}-1)+8$\\
            $\because 4$ evenly divides $3^{2\cdot k} - 1$ and $8$\\
            $\therefore 4$ evenly divides $3^{2\cdot(k+1)}-1$\\
            Therefore, for any positive integer $n$, $4$ evenly divides $3^{2n}-1$.
        \end{proof}
    \end{enumerate}
\end{homeworkProblem}

\end{document}