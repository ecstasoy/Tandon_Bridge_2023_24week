\documentclass{article}

\usepackage{fancyhdr}
\usepackage{extramarks}
\usepackage{amsmath}
\usepackage{amsthm}
\usepackage{amsfonts}
\usepackage{tikz}
\usepackage[plain]{algorithm}
\usepackage{algpseudocode}
\usepackage{xcolor}
\usepackage{enumerate}
\usepackage{amssymb}
\usepackage{graphicx}
\usepackage{float} 
\usepackage{subfigure}


\usetikzlibrary{automata,positioning}

\topmargin=-0.45in
\evensidemargin=0in
\oddsidemargin=0in
\textwidth=6.5in
\textheight=9.0in
\headsep=0.25in

\linespread{1.1}

\pagestyle{fancy}
\lhead{\hmwkAuthorName}
\chead{\hmwkClass\ (\hmwkClassInstructor\ \hmwkClassTime): \hmwkTitle}
\rhead{\firstxmark}
\lfoot{\lastxmark}
\cfoot{\thepage}

\renewcommand\headrulewidth{0.4pt}
\renewcommand\footrulewidth{0.4pt}

\setlength\parindent{0pt}

\newcommand{\enterProblemHeader}[1]{
    \nobreak\extramarks{}{Problem \arabic{#1} continued on next page\ldots}\nobreak{}
    \nobreak\extramarks{Problem \arabic{#1} (continued)}{Problem \arabic{#1} continued on next page\ldots}\nobreak{}
}

\newcommand{\exitProblemHeader}[1]{
    \nobreak\extramarks{Problem \arabic{#1} (continued)}{Problem \arabic{#1} continued on next page\ldots}\nobreak{}
    \stepcounter{#1}
    \nobreak\extramarks{Problem \arabic{#1}}{}\nobreak{}
}

\setcounter{secnumdepth}{0}
\newcounter{partCounter}
\newcounter{homeworkProblemCounter}
\setcounter{homeworkProblemCounter}{3}
\nobreak\extramarks{Problem \arabic{homeworkProblemCounter}}{}\nobreak{}

\newenvironment{homeworkProblem}[1][-1]{
    \ifnum#1>0
        \setcounter{homeworkProblemCounter}{#1}
    \fi
    \section{Problem \arabic{homeworkProblemCounter}}
    \setcounter{partCounter}{1}
    \enterProblemHeader{homeworkProblemCounter}
}{
    \exitProblemHeader{homeworkProblemCounter}
}

\newcommand{\hmwkTitle}{HW\ \#5}
\newcommand{\hmwkDueDate}{August 13, 2023}
\newcommand{\hmwkClass}{Tandon CS Bridge}
\newcommand{\hmwkClassTime}{Extended 24-week}
\newcommand{\hmwkClassInstructor}{Ratan Dey}
\newcommand{\hmwkAuthorName}{\textbf{Kunhua Huang}}
\newcommand{\hmwkNYUID}{kh4092}

\title{
    \vspace{2in}
    \textmd{\textbf{\hmwkClass:\ \hmwkTitle}}\\
    \normalsize\vspace{0.1in}\small{Due\ on\ \hmwkDueDate}\\
    \vspace{0.1in}\large{\textit{\hmwkClassInstructor\, \hmwkClassTime}}
    \vspace{3in}
}

\author{
    \vspace{0.1in}
    \textmd{\hmwkAuthorName}\\
    \text{\hmwkNYUID}
}
\date{}

\renewcommand{\part}[1]{\textbf{\large Part \Alph{partCounter}}\stepcounter{partCounter}\\}

%
% Various Helper Commands
%

% Useful for algorithms
\newcommand{\alg}[1]{\textsc{\bfseries \footnotesize #1}}

% For derivatives
\newcommand{\deriv}[1]{\frac{\mathrm{d}}{\mathrm{d}x} (#1)}

% For partial derivatives
\newcommand{\pderiv}[2]{\frac{\partial}{\partial #1} (#2)}

% Integral dx
\newcommand{\dx}{\mathrm{d}x}

% Alias for the Solution section header
\newcommand{\solution}{\textbf{\large Solution}}

\newcommand\logeq{\mathrel{\raisebox{.66pt}{:}}\Leftrightarrow}

% Probability commands: Expectation, Variance, Covariance, Bias
\newcommand{\E}{\mathrm{E}}
\newcommand{\Var}{\mathrm{Var}}
\newcommand{\Cov}{\mathrm{Cov}}
\newcommand{\Bias}{\mathrm{Bias}}

\begin{document}

\maketitle

\pagebreak

\begin{homeworkProblem}
    Solve the following questions from the Discrete Math ZyBook:
    \begin{enumerate}
        \item Excercise 4.1.3, sections b, c
        \\
        Which of the following functions are from $\mathbb{R}$ to $\mathbb{R}$? If f is a function, give its range.
        \begin{enumerate}
            \item $f(x)=1/(x^2-4)$
            \item $f(x)=\sqrt{x^2}$
        \end{enumerate}

        \textbf{Solution}
        \begin{enumerate}
            \item The range of this function is $\{x\in\mathbb{R}:\;x\neq-2\ and\ x\neq2\}$.
            \item The range of this function is from $\mathbb{R}$ to $\mathbb{R}$.
        \end{enumerate}
        \item Excercise 4.1.5, sections b, d, h, i, l
        \\
        Express the range of each function using roster notation.
        \begin{enumerate}
            \item Let $A=\{2,3,4,5\}$.\\
            $f:A\rightarrow\mathbb{Z}$ such that $f(x)=x^2$
            \item $f:\{0,1\}^5\rightarrow\mathbb{Z}$
            \item Let $A=\{1,2,3\}$.\\
            $f:A \times A\rightarrow\mathbb{Z}\times\mathbb{Z}$, where $f(x,y)=(y,x)$
            \item Let $A=\{1,2,3\}$\\
            $f:A \times A\rightarrow\mathbb{Z}\times\mathbb{Z}$, where $f(x,y)=f(x,y+1)$
            \item Let$A=\{1,2,3\}$.\\
            $f:P(A)\rightarrow P(A)$. For $X \subseteq A$, $f(X)=X-\{1\}$
        \end{enumerate}
        \textbf{Solution}
        \begin{enumerate}
            \item $\{4,9,16,25\}$
            \item $\{0,1,2,3,4,5\}$
            \item $\{(1,1),(1,2),(1,3),(2,1),(2,2),(2,3),(3,1),(3,2),(3,3)\}$
            \item $\{(1,2),(1,3),(1,4),(2,2),(2,3),(2,4),(3,2),(3,3),(3,4)\}$
            \item $\{\emptyset,\{2\},\{3\},\{1,2\},\{1,3\},\{2,3\},\{1,2,3\}\}$
        \end{enumerate}
    \end{enumerate}
\end{homeworkProblem}

\pagebreak

\begin{homeworkProblem}
    \begin{enumerate}
        \item Solve the following questions from the Discrete ZyBook:
        \begin{enumerate}
            \item Excercise 4.2.2, sections c, g, k
            \\
            For each of the functions below, indicate whether the function is onto, one-to-one, neither or both. If the function is not onto or not one-to-one, give an example showing why.
            \begin{enumerate}
                \item $f:\mathbb{Z}\rightarrow\mathbb{Z}.h(x)=x^3$
                \item $f:\mathbb{Z}\times\mathbb{Z}\rightarrow\mathbb{Z}\times\mathbb{Z}.f(x,y)=(x+1,2y)$
                \item $f:\mathbb{Z}\times\mathbb{Z}\rightarrow\mathbb{Z}\times\mathbb{Z}.f(x,y)=(\lceil\frac{x}{5}\rceil,5y-2)$
            \end{enumerate}
            \textbf{Solution}
            \begin{enumerate}
                \item This is a one-to-one and onto function.
                \item This is a one-to-one but not onto function.
                \item The function is onto but not one-to-one.
            \end{enumerate}
            \item Excercise 4.2.4, sections b, c, d, g
            \\
            For each of the functions below, indicate whether the function is onto, one-to-one, neither or both. If the function is not onto or not one-to-one, give an example showing why.
            \begin{enumerate}
                \item $f:\{0,1\}^3\rightarrow\{0,1\}^3$. The output of f is obtained by taking the input string and replacing the first bit by 1, regardless of whether the first bit is a 0 or 1.
                \item $f:\{0,1\}^3\rightarrow\{0,1\}^3$. The output of f is obtained by taking the input string and reversing the bits.
                \item $f:\{0,1\}^3\rightarrow\{0,1\}^4$. The output of f is obtained by taking the input string and adding an extra copy of the first bit to the end of the string.
                \item Let A be defined to be the set $\{1, 2, 3, 4, 5, 6, 7, 8\}$ and let $B = \{1\}$. $f: P(A) \rightarrow P(A)$. For $X \subseteq A, f(X) = X - B$.
            \end{enumerate}
            \textbf{Solution}
            \begin{enumerate}
                \item This function is one-to-one but not onto.
                \item This function is both one-to-one and onto.
                \item This function is one-to-one but not onto.
                \item This function is one-to-one but not onto.
            \end{enumerate}
        \end{enumerate}
        \item Give an example of a function from the set of integers to the set of positive integers that is:
        \begin{enumerate}
            \item one-to-one, but not onto.
            \item onto, but not one-to-one.
            \item one-to-one and onto.
            \item neither one-to-one nor onto.
        \end{enumerate}
        \textbf{Solution}
        \begin{enumerate}
            \item \[
            f(x) =
            \left\{
                \begin{array}{ lr }
                    x,\ x <= 0\\
                    x+2,\ x > 0 
                \end{array}
            \right.
            \]
            \item \[
            f(x) =
            \left\{
                \begin{array}{ lr }
                    x,\ x\ is\ odd\\
                    x/2,\ x\ is\ even 
                \end{array}
            \right.
            \]
            \item $f(x)=x$
            \item $f(x)=1$
        \end{enumerate}
    \end{enumerate}
\end{homeworkProblem}

\pagebreak

\begin{homeworkProblem}
    Solve the following questions from the Discrete Math ZyBook:
    \begin{enumerate}
        \item Excercise 4.3.2, sections c, d, g, i
        \\
        For each of the following functions, indicate whether the function has a well-defined inverse. If the inverse is well-defined, give the input/output relationship of $f^{-1}$.
        \begin{enumerate}
            \item $f:\mathbb{R}\rightarrow\mathbb{R}.\ f(x)=2x+3$
            \item Let $A$ be defined to be the set $\{1,2,3,4,5,6,7,8\}$.\\
            $f:P(A)\rightarrow\{1,2,3,4,5,6,7,8\}$.\\
            For $X\subseteq A,\ f(X)=|X|$
            \item $f:\{0,1\}^3\rightarrow \{0,1\}^3$, The output of $f$ is obtained by taking the input string and reversing the bits.
            \item $f:\mathbb{Z}\times\mathbb{Z}\rightarrow\mathbb{Z}\times\mathbb{Z},\ f(x,y)=(x+5,y-2)$
        \end{enumerate}
        \textbf{Solution}
        \begin{enumerate}
            \item The function has a well-defined inverse: $f^{-1}:\mathbb{R}\rightarrow\mathbb{R}.\ f^{-1}(y)=\frac{y-3}{2}$
            \item The function does not have a well-defined inverse.
            \item The function has a well-defined inverse: $f^{-1}:\{0,1\}^3\rightarrow \{0,1\}^3$. The output of $f$ is obtained by taking the input string and reversing the bits.
            \item The function has a well-defined inverse: $f^{-1}:\mathbb{Z}\times\mathbb{Z}\rightarrow\mathbb{Z}\times\mathbb{Z},\ f(x,y)\rightarrow(x-5,y+2)$.
        \end{enumerate}
        \item Excercise 4.4.8, sections c, d
        \\
        The domain and target set of functions f, g, and h are $\mathbb{Z}$. The functions are defined as:
        \[
            \left\{
                \begin{array}{  lr  }
                    f(x)=2x+3\\
                    g(x)=5x+7\\
                    h(x)=x^2+1
                \end{array}
            \right.    
        \]
        Give an explicit formula for each function given below.
        \begin{enumerate}
            \item f o h
            \item h o f
        \end{enumerate}

        \textbf{Solution}
        \begin{enumerate}
            \item $f(h(x))=2x^2+5$
            \item $h(f(x))=4x^2+6x+10$
        \end{enumerate}
        \item Excercise 4.4.2, section b-d
        \\
        Consider three functions f, g, and h, whose domain and target are $\mathbb{Z}$. Let:
        \[
            \left\{
                \begin{array}{  lr  }
                    f(x)=x^2\\
                    g(x)=2^x\\
                    h(x)=\lceil\frac{x}{5}\rceil
                \end{array}
            \right.    
        \]
        \begin{enumerate}
            \item Evaluate $(f\ o\ h)(52)$
            \item Evaluate $(g\ o\ h\ o\ f)(4)$
            \item Give a mathematical expression for h o f.
            \item Give a mathematical expression for f o g.
        \end{enumerate}
        \textbf{Solution}
        \begin{enumerate}
            \item $(f\ o\ h)=(\lceil\frac{x}{5}\rceil)^2$
            \\
            $(f\ o\ h)(52)=(\lceil\frac{52}{5}\rceil)^2=11^2=121$
            \item $(g\ o\ h\ o\ f)=2^{\lceil\frac{x^2}{5}\rceil}$
            \\
            $(g\ o\ h\ o\ f)(4)=2^{\lceil\frac{4^2}{5}\rceil}=2^4=16$
            \item  $(h\ o\ f)=\lceil\frac{x^2}{5}\rceil$
            \item $(f\ o\ g)=(2^x)^2=2^{2x}$
        \end{enumerate}
        \item Excercise 4.4.6, sections c-e
        \\
        Define the following functions f, g, and h:
        \[
            \left\{
                \begin{array}{  lr  }
                    f:\{0,1\}^3\rightarrow\{0,1\}^3,\ \text{The output of f is obtained by taking the input string and replacing the first bit by 1,}\\ \text{regardless of whether the first bit is a 0 or 1.}\\
                    g:\{0,1\}^3\rightarrow\{0,1\}^3,\ \text{The output of g is obtained by taking the input string and reversing the bits.}\\
                    h:\{0,1\}^3\rightarrow\{0,1\}^3,\ \text{The output of h is obtained by taking the input string x,}\\ \text{and replacing the last bit with a copy of the first bit.}
                \end{array}
            \right.    
        \]
        \begin{enumerate}
            \item What is (h o f)(010)
            \item What is the range of h o f?
            \item What is the range of g o f?
        \end{enumerate}
        \textbf{Solution}
        \begin{enumerate}
            \item $(111)$
            \item $\{101,111\}$
            \item $\{000,100,010,001,101,011,110,111\}$
        \end{enumerate}
        \item Excercise 4.4.4, sections c, d
        \\
        Let $f:X\rightarrow Y$ and $g:Y\rightarrow Z$ be two functions.
        \begin{enumerate}
            \item Is it possible that f is not one-to-one and g o f is one-to-one? Justify your answer. If the answer is "yes", give a specific example for f and g.
            \item Is it possible that g is not one-to-one and g o f is one-to-one? Justify your answer. If the answer is "yes", give a specific example for f and g.
        \end{enumerate}
        \textbf{Solution}
        \begin{enumerate}
            \item Yes. Suppose $f(x)=1$ and $g(x)=x$, then $g(f(x))$ should be a one-to-one function.
            \item No. If $g(x)$ id not a one-to-one function, then $g(f(x))$ is not a one-to-one fuction as well. Because for $g(x)$ there is always at least one element in the target that has more than one element in its domain.
        \end{enumerate}
    \end{enumerate}
\end{homeworkProblem}

\end{document}